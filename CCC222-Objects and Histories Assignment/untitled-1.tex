\documentclass{article}
\title{\textsc{CCC222: Objects and Histories Assignment}}
\author{{\bf Author}: Himanshu Sharma \hspace{50mm} {\bf Roll Number}: 1610110149 \\\\ Dept. of Electrical Engineering (Electronics and Communication Engineering)}
\date{}

\usepackage[margin=1.1in]{geometry}
\usepackage{graphicx}
\usepackage{float}
\usepackage{fourier-orns}

\begin{document}
\maketitle
The object of which I am going to talk about is a FM Radio, a very old one which was given to me as a birthday gift on my first birthday. In today's time, a dedicated radio set has become something obsolete. Now days radios are inbuilt in many smart devices, but around 1997 when I got it, at that time smart devices were just coming into the market and radio sets were more popular. The radio set which I got is not available in market now. I will first discuss about the physical dimensions of it. It is black in color and has an adjustable antenna. From its built, I can guess that it is purely analog in nature built from resistors, capacitors (condensors) and inductors. Radios found today are not objects, rather they are softwares found inside the digital devices. 

\par Judging from the object, I can say that the radio was given to me because as a child I may find it interesting, obviously, a toddler at that age; I possibly could not understand whatever it played, but the feel of the music is something that even a toddler could understand. There is an adjustable antenna on this radio which detects radio waves floating in the air. At that time, using an antenna was the only means to capture the signals but in present scenario with the emergence of better communication technologies like the Internet, the antenna technology is now shifting to satellite technology. The radio fascinated me about how we can transmit voices in the air at such long distances and made me took a career as an Electrical Engineer, possibly. I remember, that once while playing with it, I smashed the radio on the floor, by mistake and it still did not break. It reminds me of the quality of the material from which it is built. It has a mechanical display and not a digital one. There is a white stick which shows the frequency at which the radio is operating and there are no buttons on it to control it, just two wheels which when rotated can control the device. It has a handle through which it can be moved alongwith just like a suitcase does. At that time, radios were quite popular and people used to carry them along with themselves. Now days, people can't even think of doing that, basically everything is on the Internet. Radios at that time were companions just like smartphones today are. The radio which I have is just like that. It was meant to be portable so that it could be used anywhere we like. 

\par The radio which I am talking about does not hold any special history as the reader might have noticed, but it is something precious that I got as a gift. I used it quite a lot but now it is kept inside my safe at my home just like some antique object. Now it does not work at all. The chemicals inside the cells have flown out and the battery terminals are now rusted. The antenna is damaged and it moves freely instead of being rigid at a place. There is something in this radio which keeps me away from throwing it. Probably, because I don't want to throw it away. I don't usually take it out from the safe, but when this time I had to take it out for writing this essay, I was surprised to see that the handle of the radio which was used to hold it for carrying it over, was broken. I don't know how it happened but that intrigued me to ask my parents about it and to my surprise it was me who broke it when I was a kid. Probably while playing with it. I don't exactly remeber that but I know that my radio was quite frequently taken to the shops for repair. I even once opened it and removed its speakers and I was amazed to find the complex inner circuitry, as I said, the electric circuits intrigued me take a professional career in the field of electronics.

\par I remember how I used to experiment with the radio. When the signals were weak, I used to attach a very small copper wire on the top of the antenna and it usually worked. There is still a copper wire attached to the antenna and I don't know from where I grabbed that piece of copper wire. The usual source of copper wires for me was the extra telephone line wires kept in the store room. I used scissors to cut down the wires and remove the vinyl cover off them to get the pure copper extract. We had a long pole at the terrace, if I don't get the copper wire, I used to touch the antenna of the radio to that pole. The pole was made of iron and was long enough in height. Amazingly, the radio would work. I got to know that this happens because the pole was able to catch the radio waves floating in the air easily and transfer them to the antenna and from there the antenna transferred it to the radio. My radio inspired me a lot; I wouldn't say that it was the sole object of inspiration but it did made me think how things actually work.

\par I can't find any specific or interesting history associated with the radio and that's the problem with our generation, of course including me, that we take objects for granted. But whatever little I can recall, I am writing it over here. One of those memoirs that I remember is when I was around 5 or 6 years old, probably in upper kindergarten, when my mother used to keep the radio beside me in the chilly winters and we used to take sunbath in the balcony in the noons. The peaceful sound of the radio and the warmth of the sun are some of my precious memories associated with my radio. We used to listen to plays and news and old songs. There is a switch at the back of the radio which is used to switch between AM and FM stations. I had a notion that AM broadcast include old Indian songs and FM includes the latest songs, obviously I was wrong. At that time Television sets were not the only prime source of entertainment but radios were also in the picture.

\par To me, my radio is and will be an important piece of history, rather, a history that only concerns me not the general public. I think it is the oldest thing I could have found out in my home. As I said, we don't usually understand the importance of objects in our life. Our ancestors were probably good at that. With advent of new technologies and other interesting objects in our homes now days, we don't usually care about old objects until it holds some specific importance to us. It is obvious that objects have histories associated with them but I think they are even capable of creating their own histories, like my radio did. An object like \textit{the dancing girl} is something which has a history, but a recent object like a radio has both the history and the ability to create history, I suppose. Afterall what could be history in context of my radio? Something like where it was built, who made it? I don't know the answers to these question but I surely know what it meant to me. I wish those days could come back when I used to listen it with my mother on the terrace while playing around in my vacations. Those golden days are now no longer there. Now my radio rests in peace inside my safe and who knows whether it will even be used again.
\\
\begin{center}
\decoone \decoone \decoone
\end{center}
\end{document}